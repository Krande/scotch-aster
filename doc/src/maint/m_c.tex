%%%%%%%%%%%%%%%%%%%%%%%%%%%%%%%%%%%%
%                                  %
% Title   : m_c.tex                %
% Subject : Maintenance manual of  %
%           Scotch                 %
%           Code explanations      %
% Author  : Francois Pellegrini    %
%                                  %
%%%%%%%%%%%%%%%%%%%%%%%%%%%%%%%%%%%%

\section{Code explanations}
\label{sec-code}

This section explains some of the most complex algorithms implemented
in \scotch\ and \ptscotch.

\subsection{\texttt{dgraphCoarsenBuild()}}

The \texttt{dgraphCoarsenBuild()} routine creates a coarse distributed
graph from a fine distributed graph, using the result of a distributed
matching. The result of the matching is available on all MPI processes
as follows:
\begin{itemize}
\item
  \texttt{coardat.\lbt multlocnbr}: the number of local coarse
  vertices to be created;
\item
  \texttt{coardat.\lbt multloctab}: the local multinode array. For each
  local coarse vertex to be created, it contains two values. The first
  one is always positive, and represents the global number of the first
  local fine vertex to be mated. The second number can be either
  positive or negative. If it is positive, it represents the global
  number of the second local fine vertex to be mated. If it is
  negative, its opposite, minus two, represents the local edge number
  pointing to the remote vertex to be mated;
  \texttt{coardat.\lbt procgsttax}: array (restricted to ghost
  vertices only) that records on which process is located each ghost
  fine vertex.
\end{itemize}

\subsubsection{Creating the fine-to-coarse vertex array}

In order to build the coarse graph, one should create the array that
provides the coarse global vertex number for all fine vertex ends
(local and ghost). This information will be stored in the
\texttt{coardat.\lbt coargsttax} array.

Hence, a loop on local multinode data fills
\texttt{coardat.\lbt coargsttax}. The first local multinode vertex
index is always local, by nature of the matching algorithm.
If the second vertex is local too, \texttt{coardat.\lbt coargsttax} is
filled instantly. Else, a request for the global coarse vertex number
of the remote vertex is forged, in the \texttt{vsnddattab} array,
indexed by the current index \texttt{coarsndidx} extracted from the
neighbor process send index table \texttt{nsndidxtab}. Each request
comprises two numbers: the global fine number of the remote vertex for
which the coarse number is seeked, and the global number of the
coarse multinode vertex into which it will be merged.

Then, an all-to-all-v data exchange by communication takes place,
using either the \texttt{dgraph\lbt Coarsen\lbt Build\lbt Ptop()} or
\texttt{dgraph\texttt Coarsen\lbt Build\lbt Coll()} routines.  Apart
from the type of communication they implement (either point-to-point
or collective), these routines do the same task: they process the
pairs of values sent from the \texttt{vsnddattab} array. For each pair
(the order of processing is irrelevant), the \texttt{coargsttax} array
of the receiving process is filled-in with the global multinode value
of the remotely mated vertex. Hence, at the end of this phase, all
processes have a fully valid local part of the \texttt{coargsttax}
array; no value should remain negative (as set by default). Also, the
\texttt{nrcvidxtab} array is filled, for each neighbor process, of the
number of data it has sent. This number is preserved, as it will serve
to determine the number of adjacency data to be sent back to each
neighbor process.

Then, data arrays for sending edge adjacency are filled-in. The
\texttt{ercvdsptab} and \texttt{ercvcnttab} arrays, of size
\texttt{procglbnbr}, are computed according to the data stored in
\texttt{coardat.\lbt dcntglbtab}, regarding the number of vertex- and
edge-related data to exchange.

By way of a call to \texttt{dgraphHaloSync()}, the ghost data of the
\texttt{coargsttax} array are exchanged.

Then, \texttt{edgelocnbr}, an upper bound on the number of local
edges, as well as \texttt{ercvdatsiz} and \texttt{esnddatsiz}, the
edge receive and send array sizes, respectively.

Then, all data arrays for the coarse graph are allocated, plus the
main adjacency send array \texttt{esnddsptab}, its receive counterpart
\texttt{ercvdattab}, and the index send arrays \texttt{esnddsptab} and
\texttt{esndcnttab}, among others.

Then, adjacency send arrays are filled-in. This is done by performing
a loop on all processes, within which only neighbor processes are
actually considered, while index data in \texttt{esnddsptab} and
\texttt{esndcnttab} is set to $0$ for non-neighbor processes. For each
neighbor process, and for each vertex local which was remotely mated
by this neighbor process, the vertex degree is written in the
\texttt{esnddsptab} array, plus optionally its load, plus the edge
data for each of its neighbor vertices: the coarse number of its end,
obtained through the \texttt{coargsttax} array, plus optionally the
edge load. At this stage, two edges linking to the same coarse
multinode will not be merged together, because this would have
required a hash table on the send side. The actual merging will be
performed once, on the receive side, in the next stage of the
algorithm.
