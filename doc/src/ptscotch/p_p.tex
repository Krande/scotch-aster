%%%%%%%%%%%%%%%%%%%%%%%%%%%%%%%%%%%%
%                                  %
% Titre  : p_p.tex                 %
% Sujet  : Manuel de l'utilisateur %
%          du projet 'PT-Scotch'   %
%          Programmes 6.0          %
% Auteur : Francois Pellegrini     %
%                                  %
%%%%%%%%%%%%%%%%%%%%%%%%%%%%%%%%%%%%

\section{Programs}
\label{sec-prog}

\subsection{Invocation}

All of the programs comprised in the \scotch\ and
\ptscotch\ distributions have been designed to run in
command-line mode without any interactive prompting,
so that they can be called easily from other programs by means of
``\mbox{\tt system$\,$()}'' or ``\mbox{\tt popen$\,$()}'' system calls, or be
piped together on a single shell command line. In order to facilitate this,
whenever a stream name is asked for (either on input or output), the user may
put a single ``{\tt -}'' to indicate standard input or output.
Moreover, programs read their input in the same order as stream names are
given in the command line. It allows them to read all their data from a
single stream (usually the standard input), provided that these data are
ordered properly.

A brief on-line help is provided with all the programs. To get this
help, use the ``{\tt -h}'' option after the program name. The case of
option letters is not significant, except when both the lower and
upper cases of a letter have different meanings. When passing
parameters to the programs, only the order of file names is
significant; options can be put anywhere in the command line, in any
order. Examples of use of the different programs
of the \ptscotch\ project are provided in section~\ref{sec-examples}.

Error messages are standardized, but may not be fully explanatory.
However, most of the errors you may run into should be related to file
formats, and located in ``\mbox{\tt \ldots Load}'' routines.
In this case, compare your data formats with the definitions
given in section~\ref{sec-file}, and use the {\tt dgtst}
% and {\tt dmtst}
program of the \ptscotch\ distribution to check the consistency of
your distributed source graphs.
% and meshes.
\\

According to your MPI environment, you may either run the programs
directly, or else have to invoke them by means of a command such as
{\tt mpirun}. Check your local MPI documentation to see how to
specify the number of processors on which to run them.

\subsection{File names}
\label{sec-prog-filename}

\subsubsection{Sequential and parallel file opening}

The programs of the \ptscotch\ distribution can handle either
the classical centralized \scotch\ graph files, or the
distributed \ptscotch\ graph files described in
section~\ref{sec-file-dsgraph}.

In order to tell whether programs should read from, or write to, a single
file located on only one processor, or to multiple instances of the same
file on all of the processors, or else to distinct files on each of the
processors, a special grammar has been designed, which is based on the
``{\tt \%}'' escape character. Four such escape sequences are defined,
which are interpreted independently on every processor, prior to
file opening. By default, when a filename is provided, it is assumed
that the file is to be opened on only one of the processors, called
the root processor, which is usually process $0$ of the communicator
within which the program is run. Using any of the first three
escape sequences below will instruct programs to open in parallel a file
of name equal to the interpreted filename, on every processor on which they
are run. 
\begin{itemize}
\iteme[{\tt \%p}]
Replaced by the number of processes in the global communicator in
which the program is run. Leads to parallel opening.
\iteme[{\tt \%r}]
Replaced on each process running the program by the rank of this
process in the global communicator. Leads to parallel opening.
\iteme[{\tt \%-}]
Discarded, but leads to parallel opening. This sequence is mainly used to
instruct programs to open on every processor a file of identical
name. The opened files can be, according whether the given path leads
to a shared directory or to directories that are local to each
processor, either to the opening of multiple instances of the same
file, or to the opening of distinct files which may each have a
different content, respectively (but in this latter case it is much
recommended to identify files by means of the ``{\tt \%r}'' sequence).
\iteme[{\tt \%\%}]
Replaced by a single ``{\tt \%}'' character. File names using this
escape sequence are not considered for parallel opening, unless one or
several of the three other escape sequences are also present.
\end{itemize}
For instance, filename ``{\tt brol}'' will lead to the
opening of file ``{\tt brol}'' on the root processor only, filename
``{\tt \%-brol}'' (or even ``{\tt br\%-ol}'') will lead to the
parallel opening of files called ``{\tt brol}'' on every processor,
and filename ``{\tt brol\%p-\%r}'' will lead to the opening of files
``{\tt brol2-0}'' and ``{\tt brol2-1}'', respectively, on each of the
two processors on which which would run a program of the
\ptscotch\ distribution.

\subsubsection{Using compressed files}
\label{sec-prog-compressed}

Starting from version 5.0.6, \scotch\ allows users to provide and
retrieve data in compressed form. Since this feature requires that
the compression and decompression tasks run in the same time as data
is read or written, it can only be done on systems which support
multi-threading (Posix threads) or multi-processing (by means of
{\tt fork} system calls).

To determine if a stream has to be handled in compressed form,
\scotch\ checks its extension. If it is ``{\tt .gz}'' ({\tt gzip}
format), ``{\tt .bz2}'' ({\tt bzip2} format) or ``{\tt .lzma}''
({\tt lzma} format), the stream is assumed to be compressed according
to the corresponding format. A filter task will then be used to process
it accordingly if the format is implemented in \scotch\ and enabled on
your system.

To date, data can be read and written in {\tt bzip2} and {\tt gzip}
formats, and can also be read in the {\tt lzma} format. Since the
compression ratio of {\tt lzma} on \scotch\ graphs is $30\%$ better
than the one of {\tt gzip} and {\tt bzip2} (which are almost
equivalent in this case), the {\tt lzma} format is a very good choice
for handling very large graphs. To see how to enable compressed data
handling in \scotch, please refer to Section~\ref{sec-install}.
\\

When the compressed format allows it, several files can be provided on
the same stream, and be uncompressed on the fly. For instance, the
command ``{\tt cat brol.grf.gz brol.xyz.gz | gout -.gz -.gz -Mn -
brol.iv}'' concatenates the topology and geometry data of some graph
{\tt brol} and feed them as a single compressed stream to the standard
input of program {\tt gout}, hence the ''{\tt -.gz}'' to indicate a
compressed standard stream.

\subsection{Description}

\subsubsection{{\tt dgmap} / {\tt dgpart}}
\label{sec-prog-dgmap}

\begin{itemize}
\progsyn
{\tt dgmap} [{\it input\_graph\_file} [{\it input\_\lbt target\_\lbt file} [{\it output\_\lbt mapping\_\lbt file} [{\it output\_\lbt log\_\lbt file}]]]] {\it options}\\
~\\
{\tt dgpart} {\it number\_\lbt of\_\lbt parts} [{\it input\_graph\_file} [{\it output\_\lbt mapping\_\lbt file} [{\it output\_\lbt log\_\lbt file}]]] {\it options}

\progdes

The {\tt dgmap} program is the parallel static mapper. It uses a
static mapping strategy to compute a mapping of the given source graph
to the given target architecture. The implemented algorithms aim at
assigning source graph vertices to target vertices such that every
target vertex receives a set of source vertices of summed weight
proportional to the relative weight of the target vertex in the target
architecture, and such that the communication cost function $f_C$ is
minimized (see Section~\ref{sec-algo-cost} for the definition and
rationale of this cost function).

Since its main purpose is to provide mappings that exhibit high
concurrency for communication minimization in the mapped application,
it comprises a parallel implementation of the dual recursive
bipartitioning algorithm~\cite{pell94a}, as well as all of the
sequential static mapping methods used by its sequential counterpart
{\tt gmap}, to be used on subgraphs located on single processors.

{\tt dgpart} is a simplified interface to {\tt dgmap}, which performs
graph partitioning instead of static mapping. Consequently, the
desired number of parts has to be provided, in lieu of the target
architecture.

The {\tt -b} and {\tt -c} options allow the user to set preferences on
the behavior of the mapping strategy which is used by default. The
{\tt -m} option allows the user to define a custom mapping strategy.

The {\it input\_graph\_file} filename can refer either to a
centralized or to a distributed graph, according to the semantics
defined in Section~\ref{sec-prog-filename}. The mapping file must be
a centralized file.

\progopt\\*
Since the program is devoted to experimental studies, it has many
optional parameters, used to test various execution modes. Values
set by default will give best results in most cases.
\begin{itemize}
\iteme[{\tt -b}{\it rat}]
Set the maximum load imbalance ratio to \textit{rat}, which should
be a value comprised between $0$ and $1$. This option can be used in
conjunction with option \texttt{-c}, but is incompatible with option
\texttt{-m}.
\iteme[{\tt -c}{\it flags}]
Tune the default mapping strategy according to the given preference
flags. Some of these flags are antagonistic, while others can be
combined. See Section~\ref{sec-lib-format-strat-default} for more
information. The currently available flags are the following.
\begin{itemize}
\iteme[{\tt b}]
Enforce load balance as much as possible.
\iteme[{\tt q}]
Privilege quality over speed. This is the default behavior.
\iteme[{\tt s}]
Privilege speed over quality.
\iteme[{\tt t}]
Use only safe methods in the strategy.
\iteme[{\tt x}]
Favor scalability.
\end{itemize}
This option can be used in conjunction with option \texttt{-b}, but is
incompatible with option \texttt{-m}.
The resulting strategy string can be displayed by means
of the {\tt -vs} option.
\iteme[{\tt -h}]
Display the program synopsis.
\iteme[{{\tt -m}{\it strat}}]
Apply parallel static mapping strategy {\it strat}. The format of parallel
mapping strategies is defined in section~\ref{sec-lib-format-pmap}.
This option is incompatible with options \texttt{-b} and
\texttt{-c}.
\iteme[{\tt -r}{\it num}]
Set the number of the root process which will be used for centralized
file accesses. Set to $0$ by default.
\iteme[{\tt -s}{\it obj}]
Mask source edge and vertex weights. This option allows the user to
``unweight'' weighted source graphs by removing weights from edges and
vertices at loading time. {\it obj\/} may contain several of the following
switches.
\begin{itemize}
\iteme[{\tt e}]
Remove edge weights, if any.
\iteme[{\tt v}]
Remove vertex weights, if any.
\end{itemize}
\iteme[{\tt -V}]
Print the program version and copyright.
\iteme[{\tt -v}{\it verb}]
Set verbose mode to {\it verb}, which may contain several of the following
switches.
%For a detailed description of the data displayed, please
%refer to the manual page of {\tt dgmtst} below.
\begin{itemize}
\iteme[{\tt a}]
Memory allocation information.
\iteme[{\tt m}]
Mapping information, similar to the one displayed by the {\tt gmtst}
program of the sequential \scotch\ distribution.
\iteme[{\tt s}]
Strategy information. This parameter displays the default mapping
strategy used by {\tt gmap}.
\iteme[{\tt t}]
Timing information.
\end{itemize}
\end{itemize}
\end{itemize}

\subsubsection{{\tt dgord}}

\begin{itemize}
\progsyn
{\tt dgord} [{\it input\_graph\_file} [{\it output\_ordering\_file} [{\it output\_log\_file}]]] {\it options}

\progdes

The {\tt dgord} program is the parallel sparse matrix block
orderer. It uses an ordering strategy to compute block orderings of
sparse matrices represented as source graphs, whose vertex weights
indicate the number of DOFs per node (if this number is non
homogeneous) and whose edges are unweighted, in order to minimize
fill-in and operation count.

Since its main purpose is to provide orderings that exhibit high
concurrency for parallel block factorization, it comprises a parallel
nested dissection method~\cite{geli81}, but sequential
classical~\cite{liu-85} and state-of-the-art~\cite{peroam00a}
minimum degree algorithms are implemented as well, to be used on
subgraphs located on single processors.

Ordering methods can be combined by means of selection, grouping, and
condition operators, so as to define ordering strategies, which can be
passed to the program by means of the {\tt -o} option. The {\tt -c}
option allows the user to set preferences on the behavior of the
ordering strategy which is used by default.

The {\it input\_graph\_file} filename can refer either to a
centralized or to a distributed graph, according to the semantics
defined in Section~\ref{sec-prog-filename}. The ordering file must be
a centralized file.

\progopt\\*
Since the program is devoted to experimental studies, it has many
optional parameters, used to test various execution modes. Values
set by default will give best results in most cases.
\begin{itemize}
\iteme[{\tt -c}{\it flags}]
Tune the default ordering strategy according to the given preference
flags. Some of these flags are antagonistic, while others can be
combined. See Section~\ref{sec-lib-format-strat-default} for more
information. The resulting strategy string can be displayed by means
of the {\tt -vs} option.
\begin{itemize}
\iteme[{\tt b}]
Enforce load balance as much as possible.
\iteme[{\tt q}]
Privilege quality over speed. This is the default behavior.
\iteme[{\tt s}]
Privilege speed over quality.
\iteme[{\tt t}]
Use only safe methods in the strategy.
\iteme[{\tt x}]
Favor scalability.
\end{itemize}
\iteme[{\tt -h}]
Display the program synopsis.
\iteme[{\tt -m}{\it output\_mapping\_file}]
Write to {\it output\_mapping\_file\/} the mapping of graph vertices to
column blocks. All of the separators and leaves produced by the nested
dissection method are considered as distinct column blocks, which may
be in turn split by the ordering methods that are applied to them.
Distinct integer numbers are associated with each of the column blocks,
such that the number of a block is always greater than the ones of its
predecessors in the elimination process, that is, its descendants in
the elimination tree.
The structure of mapping files is described in detail in the relevant
section of the {\it\scotch\ User's Guide}~\scotchcitesuser.

When the geometry of the graph is available, this mapping file may be
processed by program {\tt gout} to display the vertex separators and
supervariable amalgamations that have been computed.
\iteme[{{\tt -o}{\it strat}}]
Apply parallel ordering strategy {\it strat}. The format of parallel
ordering strategies is defined in section~\ref{sec-lib-format-pord}.
\iteme[{\tt -r}{\it num}]
Set the number of the root process which will be used for centralized
file accesses. Set to $0$ by default.
\iteme[{\tt -t}{\it output\_tree\_file}]
Write to {\it output\_tree\_file\/} the structure of the separator
tree. The data that is written resembles much the one of a mapping
file: after a first line that contains the number of lines to follow,
there are that many lines of mapping pairs, which associate an integer
number with every graph vertex index. This integer number is the
number of the column block which is the parent of the column block to
which the vertex belongs, or $-1$ if the column block to which the
vertex belongs is a root of the separator tree (there can be several
roots, if the graph is disconnected).

Combined to the column block mapping data produced by option {\tt -m},
the tree structure allows one to rebuild the separator tree.
\iteme[{\tt -V}]
Print the program version and copyright.
\iteme[{\tt -v}{\it verb}]
Set verbose mode to {\it verb}, which may contain several of the following
switches.
%For a detailed description of the data displayed, please
%refer to the manual page of {\tt gotst}.
\begin{itemize}
\iteme[{\tt a}]
Memory allocation information.
\iteme[{\tt s}]
Strategy information. This parameter displays the default parallel
ordering strategy used by {\tt dgord}.
\iteme[{\tt t}]
Timing information.
\end{itemize}
\end{itemize}
\end{itemize}

\subsubsection{{\tt dgpart}}

\begin{itemize}
\progsyn
{\tt dgpart} [{\it number\_of\_parts} [{\it input\_\lbt graph\_\lbt file} [{\it output\_\lbt mapping\_\lbt file} [{\it output\_\lbt log\_\lbt file}]]]] {\it options}

\progdes

The {\tt dgpart} program is the parallel graph partitioner. It is
in fact a shortcut for the {\tt dgmap} program, where the number of
parts is turned into a complete graph with same number of vertices
which is passed to the static mapping routine.

Save for the {\it number\_of\_parts} parameter which replaces the {\it
input\_target\_file}, the parameters of {\tt dgpart} are identical to
the ones of {\tt dgmap}. Please refer to its manual page, in
Section~\ref{sec-prog-dgmap}, for a description of all of the
available options.
\end{itemize}

\subsubsection{{\tt dgscat} / {\tt gscat}}

\begin{itemize}
\progsyn
{\tt dgscat} [{\it input\_graph\_file} [{\it output\_graph\_file}]] {\it options}

\progdes

The {\tt dgscat} program creates a distributed source graph, in the
\scotch\ distributed graph format, from the given centralized source
graph file.

The {\it input\_graph\_file} filename should therefore refer to a
centralized graph, while {\it output\_graph\_file} must refer to a
distributed graph, according to the semantics defined in
Section~\ref{sec-prog-filename}.

{\tt dgscat} has a sequential counterpart, called {\tt gscat}. The
latter operates by processing the source graph file on the fly, and
does not perform any consistency checking on the output it produces.

\progopt\\[-1em]
\begin{itemize}
\iteme[{\tt -c}]
Check the consistency of the distributed graph at the end of the
graph loading phase.
\iteme[{\tt -h}]
Display the program synopsis.
\iteme[{\tt -i}{\it nbr}]
For gscat only. Create an imbalanced distributed graph, in which the
distributed graph files for all processes will receive \textit{nbr}
vertices of the graph, save for the file for the last process, which
will receive all the remaining vertices.
\iteme[{\tt -r}{\it num}]
Set the number of the root process which will be used for centralized
file accesses. Set to $0$ by default.
\iteme[{\tt -V}]
Print the program version and copyright.
\end{itemize}
\end{itemize}

\subsubsection{{\tt dgtst}}

\begin{itemize}
\progsyn
{\tt dgtst} [{\it input\_graph\_file} [{\it output\_data\_file}]] {\it options}

\progdes

The program {\tt dgtst} is the source graph tester. It checks the
consistency of the input source graph structure (matching of arcs,
number of vertices and edges, etc\@.), and gives some statistics
regarding edge weights, vertex weights, and vertex degrees.

It produces the same results as the {\tt gtst} program of the
\scotch\ sequential distribution.

\progopt
\begin{itemize}
\iteme[{\tt -h}]
Display the program synopsis.
\iteme[{\tt -r}{\it num}]
Set the number of the root process which will be used for centralized
file accesses. Set to $0$ by default.
\iteme[{\tt -V}]
Print the program version and copyright.
\end{itemize}
\end{itemize}
